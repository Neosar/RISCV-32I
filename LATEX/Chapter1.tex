%\documentclass{report}



%\usepackage{minitoc} % for contents 

%\begin{document}
%\dominitoc[n]
%\nomtcpagenumbers

\chapterfont{\raggedleft}
\chapter{Introduction}

	\minitoc 
	\vspace{5mm}
	RISC-V is an open-source hardware instruction set architecture (ISA) based on established reduced instruction set computer (RISC) principles. The project began in 2010 at the University of California, Berkeley. Nowadays, well known enterprises in the hardware sector like NVIDIA and Western Digital have announced a plan to start using RISC-V processors in their future products.
	\section{Motivation}

	\label{sec:Motivation}
	
	In the 21st century the use of processors in smartphones, tablet computers and Android/iOS devices in general is following the RISC style architecture. Thats why we began looking into RISC and not other architectures (e.g. CISC) to begin with. Being fascinated by the juvenileness of this specific ISA we were thrilled to do some research and start working with it. 
	Also, we would like to see how $VHDL$ would stand up to the challenge since most architectures we could find (e.g. $BOOM$) were developed using CHISEL, a Scala variant hardware description language.
	
	\section{Development}
	\label{sec:Dev}
	The whole project was developed hierarchically using $VHDL$ and every part was designed so that it could be synthesizable. In fact, most of the modules we created for the needs of the project were also tested on Altera's Cyclone II - DE 2 FPGA Board. Concerning the evaluation methods used, we ran various simulations using ModelSim and Quartus-II's embedded simulator. Finally the complete design was successfully tested using the official tests designated for this ISA ($RV32I$).
	\section{Report Structure}
	\label{sec:Structure}
	
	What follows is a brief technical introduction to the "world" of RISC-V and the capabilities that come with it. Then there will be an analytical presentation of my design and implementation which is (as mentioned before) the single-core RISC-V 32I (Base ISA). Finally there will be a chapter dedicated to how the full design was tested, how some issues found while testing were resolved and then, there will be a few words about some future work and some improvements that can be done.
	
	
	
	
	
	
%\end{document}